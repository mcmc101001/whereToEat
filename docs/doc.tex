\documentclass{article}
\usepackage{listings}
\usepackage{hyperref}

\hypersetup{
    colorlinks=true,
    linkcolor=blue,
    urlcolor=cyan,
    }

\begin{document}
\section{Introduction}
I decided to use LaTeX instead of markdown to generate the documentation to brush up on my LaTeX skill.

\section{Tech}
\subsection{Backend}
For the backend, I used FastAPI as the main framework, with Pydantic as the data validation library.

\subsection{Frontend}
For the frontend, I used Vue.js as the main framework, with TailwindCSS for styling. Pinia was used for state management, with local storage for persistence. I also made the application into a Progressive Web App (PWA) using a Vite plugin.

\section{Deployment}
I used Docker to containerize the FastAPI backend, and deployed it on DigitalOcean. I spent lots of time trying to troubleshoot the deployment before realising a simple mistake: DigitalOcean required the port to be 8080.


I also learnt to be more mindful with dev vs production environment. Initially, my deployment worked and then it stopped working all of the sudden. This was because I forgot to update the CORS origins in my FastAPI backend to include the production URL (and other shenanigans regarding a `/').

\pagebreak

\section{How to run}
\subsection{Backend}
Start by creating a virtual environment and installing the dependencies.

\begin{lstlisting}[language=bash]
  python -m venv venv
  env\Scripts\Activate.ps1
  pip install -r requirements.txt
\end{lstlisting}

Alternatively, you can use the latest version of the \href{https://hub.docker.com/repository/docker/mcmc101001/where-to-eat-backend/general}{Docker Image}. Either ways, remember to set the environment variables for the Google Maps API key.

\subsection{Frontend}
Instal the dependencies and run the dev server.

\begin{lstlisting}[language=bash]
  npm i
  npm run dev
\end{lstlisting}

\end{document}

